% Created 2013-01-28 Mon 22:41
\documentclass[11pt]{article}
\usepackage{graphicx}
\usepackage{geometry}
\usepackage{amsmath}
\usepackage[pdftex]{hyperref}
\usepackage[font=small,labelfont=bf]{caption}
\geometry{a4paper, textwidth=6.5in, textheight=10in, marginparsep=7pt, marginparwidth=.6in}

\title{Data Mining 2013: Project Report}
\author{marinah@student.ethz.ch\\ uvictor@student.ethz.ch\\}
\date{\today}

\begin{document}
\maketitle

\section{Approximate retrieval - Locality Sensitive Hashing}
\begin{enumerate}
\item How was your choice of rows and bands motivated? How did you search for the
best parameters? \\ \\
\textbf{Answer}:

\item Conceptually, what would you have to change if you were asked to design an image
  retrieval system that you can query for similar images given a large image
  dataset? \\

\textbf{Answer}:

\end{enumerate}

\section{Large-scale Supervised Learning}

\begin{enumerate}
\item Which algorithms did you consider? Which one did you choose for the
  final submission and why? \\ \\
\textbf{Answer}:

First we considered using Online convex programming with training samples picked at random / Stochastic Gradient Descent. We also implemented and run
the PEGASOS algorithm. 

The final solution was using OCP/SGD because it was the one that obtained better scores for our submissions. 

Unfortunately, after the deadline we realised there was a bug in the code that prevented both PEGASOS and OCP/SGD from running correctly. The problem
was the random shuffling of the order of training samples was not done correctly. This caused problems also in parameter selection.

\item How did you select the parameters for your model? Which are the
  most important parameters of your model? \\ \\
\textbf{Answer}:

We considered three parameters for the OCP/SGD solution: $K, Lambda$ and the learning rate $\eta$.

We tried to vary in a grid search manner possible values for the parameters, taking into account all possible combinations with: $K \in \{32, 64, 120\}, Lambda \in \{0.03, 0.1, 0.3\}$ and $\eta \in \{0.03, 0.1, 0.3\}$. We used cross-validation to determine the best combination of parameter values.

\end{enumerate}

\section{Recommender Systems}

\begin{enumerate}
\item Which algorithm did you implement? What was your motivation? \\ \\
\textbf{Answer}:

\item How did you select the parameters for your model? \\ \\
\textbf{Answer}:

\item Does the performance measured in CTR increase monotonically during the
execution of your algorithm? Why? \\ \\
\textbf{Answer}:

\end{enumerate}

\end{document}
